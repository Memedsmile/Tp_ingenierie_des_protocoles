\subsection{Question 1}

\begin{figure}[H]
  \centering
  \begin{tikzpicture}
    % Liste des places
    \draw (-9,0) node[below left = 2pt] {$O_7a$};
    \node[draw,circle,scale=2] (o7a) at (-9, 0) {};
    \draw (-7,0) node[above left = 2pt] {$O_7b$};
    \node[draw,circle,scale=2] (o7b) at (-7, 0) {};
    \draw (-5, -7) node[below right = 2pt] {$O_6b$};
    \node[draw,circle,scale=2] (o6b) at (-5, -7) {};
    \draw (-5,-9) node[above right = 2pt] {$O_6a$};
    \node[draw,circle,scale=2] (o6a) at (-5, -9) {};
    \draw (-5,-3) node[below right = 2pt] {$E_7$};
    \node[draw,circle,scale=2] (e7) at (-5, -3) {};
    \draw (-5,-1) node[below right = 2pt] {$E_1$};
    \node[draw,circle,scale=2] (e1) at (-5, -1) {};
    \draw (-5,7) node[above right = 2pt] {$O_1b$};
    \node[draw,circle,scale=2] (o1b) at (-5, 7) {};
    \draw (-5,9) node[above right = 2pt] {$O_1a$};
    \node[draw,circle,scale=2] (o6a) at (-5, 9) {};
    \draw (-3,-1) node[below right = 2pt] {$E_2$};
    \node[draw,circle,scale=2] (e2) at (-3, -1) {};
    \draw (-3,-3) node[below right = 2pt] {$E_6$};
    \node[draw,circle,scale=2] (e6) at (-3, -3) {};
    \draw (-1, -7) node[below right = 2pt] {$O_5b$};
    \node[draw,circle,scale=2] (o5b) at (-1, -7) {};
    \draw (-1,-9) node[above right = 2pt] {$O_5a$};
    \node[draw,circle,scale=2] (o5a) at (-1, -9) {};
    \draw (-1,-3) node[below right = 2pt] {$E_5$};
    \node[draw,circle,scale=2] (e5) at (-1, -3) {};
    \draw (-1,-1) node[below right = 2pt] {$E_3$};
    \node[draw,circle,scale=2] (e3) at (-1, -1) {};
    \draw (-1,7) node[above right = 2pt] {$O_1b$};
    \node[draw,circle,scale=2] (o1b) at (-1, 7) {};
    \draw (-1,9) node[above right = 2pt] {$O_2a$};
    \node[draw,circle,scale=2] (o2a) at (-1, 9) {};
    \draw (1,0) node[above right = 2pt] {$E_4$};
    \node[draw,circle,scale=2] (e4) at (1, 0) {};
    \draw (3,2) node[above right = 2pt] {$O_3b$};
    \node[draw,circle,scale=2] (e4) at (3, 2) {};
    \draw (3,-2) node[above right = 2pt] {$0_4b$};
    \node[draw,circle,scale=2] (e4) at (3, -2) {};
    \draw (5,2) node[above right = 2pt] {$O_3a$};
    \node[draw,circle,scale=2] (e4) at (5, 2) {};
    \draw (5,-2) node[above right = 2pt] {$0_4a$};
    \node[draw,circle,scale=2] (e4) at (5, -2) {};


    % Liste des transitions
    \draw (-6,6) node[below = 10pt] {$t_1$};
    \node[draw,rectangle,yscale=4] (t1) at (-6, 6) {};
    \draw (-3,2) node[below = 10pt] {$t_2$};
    \node[draw,rectangle,yscale=4] (t2) at (-3, 2) {};
    \draw (-3,0) node[below = 10pt] {$t_3$};
    \node[draw,rectangle,yscale=4] (t3) at (-3, 0) {};
    \draw (3,2) node[below = 10pt] {$t_4$};
    \node[draw,rectangle,yscale=4] (t4) at (3, 2) {};
    \draw (3,0) node[below = 10pt] {$t_5$};
    \node[draw,rectangle,yscale=4] (t5) at (3, 0) {};
    \draw (0,-4) node[below = 10pt] {$t_6$};
    \node[draw,rectangle,yscale=4] (t6) at (0, -4) {};
    \draw (3,-2) node[below = 10pt] {$t_7$};
    \node[draw,rectangle,yscale=4] (t7) at (3, -2) {};
    \draw (3,-4) node[below = 10pt] {$t_8$};
    \node[draw,rectangle,yscale=4] (t8) at (3, -4) {};
    \draw (3,4) node[below = 10pt] {$t_9$};
    \node[draw,rectangle,yscale=4] (t9) at (3, 4) {};
    \draw (0,6) node[below = 10pt] {$t_10$};
    \node[draw,rectangle,yscale=4] (t10) at (0, 6) {};



    % Liste des arcs
    \draw[->,>=latex] (po) to[out=135,in=-135] (t1);
    \draw[->,>=latex] (t1) to[out=-45,in=45] (po);
    \draw[->,>=latex] (po) -- (t2);
    \draw[->,>=latex] (t2) -- (lettre);
    \draw[->,>=latex] (po) -- (t3);
    \draw[->,>=latex] (t3) -- (pf);
    \draw[->,>=latex] (lettre) -- (t4);
    \draw[->,>=latex] (t4) -- (operateur);
    \draw[->,>=latex] (lettre) -- (t5);
    \draw[->,>=latex] (t5) -- (pf);
    \draw[->,>=latex] (pf) to[out=-45,in=45] (t6);
    \draw[->,>=latex] (t6) to[out=135,in=-135] (pf);
    \draw[->,>=latex] (pf) -- (t7);
    \draw[->,>=latex] (t7) -- (operateur);
    \draw[->,>=latex] (pf) -- (t8);
    \draw[->,>=latex] (t8) -- (EOF);
    \draw[->,>=latex] (operateur) -- (t9);
    \draw[->,>=latex] (t9) -- (lettre);
    \draw[->,>=latex] (operateur) to[out=90,in=0] (t10);
    \draw[->,>=latex] (t10) -- (po);
    %\draw[->,>=latex] (t6) to[out=-90,in=-90] (P1);

    % Marquage
    \draw [fill](-6,4) circle (0.1) ;
  \end{tikzpicture}
  \caption{Réseau de petri associé à une expression complètement parenthésée} \label{fig:M1}
\end{figure}