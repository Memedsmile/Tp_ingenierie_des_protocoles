\subsection{Question 1}
 Le protocole peut être décrit par le réseau suivant : \\

\begin{figure}[H]
  \centering
  \begin{tikzpicture}

    % Liste des places
    \draw (0,9) node[below right = 4pt] {$P_1$};
    \node[draw,circle,scale=2] (P1) at (0, 9) {};
    \draw (0,7) node[below right = 4pt] {$P_2$};
    \node[draw,circle,scale=2] (P2) at (0,7) {};
    \draw (0,5) node[below right = 4pt] {$P_3$};
    \node[draw,circle,scale=2] (P3) at (0, 5) {};
    \draw (-2,3) node[below right = 4pt] {$P_4$};
    \node[draw,circle,scale=2] (P4) at (-2,3) {};
    \draw (-2,1) node[below left = 4pt] {$P_5$};
    \node[draw,circle,scale=2] (P5) at (-2, 1) {};
     \draw (2,3) node[below right = 4pt] {$P_6$};
    \node[draw,circle,scale=2] (P6) at (2, 3) {};
    \draw (2,1) node[below right = 4pt] {$P_7$};
    \node[draw,circle,scale=2] (P7) at (2,1) {};
    \draw (4,1) node[below right = 4pt] {$P_8$};
    \node[draw,circle,scale=2] (P8) at (4, 1) {};
    \draw (2,-1) node[below left = 4pt] {$P_9$};
    \node[draw,circle,scale=2] (P9) at (2,-1) {};

     % Liste des transitions
    \draw (0,10) node[below right = 4pt] {$t_1$};
    \node[draw,rectangle,yscale=4] (t1) at (0, 10) {};
    \draw (0,8) node[below right = 4pt] {$t_2$};
    \node[draw,rectangle,yscale=4] (t2) at (0, 8) {};
    \draw (0,6) node[below right= 4pt] {$t_3$};
    \node[draw,rectangle,yscale=4] (t3) at (0, 6) {};
    \draw (-2,4) node[below right = 4pt] {$t_4$};
    \node[draw,rectangle,yscale=4] (t4) at (-2, 4) {};
    \draw (-2,2) node[below right = 4pt] {$t_5$};
    \node[draw,rectangle,yscale=4] (t5) at (-2, 2) {};
    \draw (2,4) node[below right = 3pt] {$t_6$};
    \node[draw,rectangle,yscale=4] (t6) at (2, 4) {};
    \draw (2,2) node[below right = 3pt] {$t_7$};
    \node[draw,rectangle,yscale=4] (t7) at (2, 2) {};
    \draw (4,2) node[below right = 3pt] {$t_8$};
    \node[draw,rectangle,yscale=4] (t8) at (4, 2) {};
    \draw (2,0) node[below right = 3pt] {$t_9$};
    \node[draw,rectangle,yscale=4] (t9) at (2, 0) {};
    \draw (2,-2) node[below right = 3pt] {$t_10$};
    \node[draw,rectangle,yscale=4] (t10) at (2, -2) {};

    % Liste des arcs
    \draw[->,>=latex] (t1) -- (P1);
    \draw[->,>=latex] (P1) -- (t2);
    \draw[->,>=latex] (t2) -- (P2);
    \draw[->,>=latex] (P2) -- (t3);
    \draw[->,>=latex] (t3) -- (P3);
    \draw[->,>=latex] (P3) -- (t4);
    \draw[->,>=latex] (t4) -- (P4);
    \draw[->,>=latex] (P4) -- (t5);
    \draw[->,>=latex] (t5) -- (P5);
    \draw[->,>=latex] (P5) -- (t10);
    \draw[->,>=latex] (P3) -- (t6);
    \draw[->,>=latex] (t6) -- (P6);
    \draw[->,>=latex] (P6) -- (t7);
    \draw[->,>=latex] (P6) -- (t8);
    \draw[->,>=latex] (t7) -- (P7);
    \draw[->,>=latex] (t8) -- (P8);
    \draw[->,>=latex] (P7) -- (t9);
    \draw[->,>=latex] (t9) -- (P9);
    \draw[->,>=latex] (P9) -- (t10);
    \draw[->,>=latex] (P8) -- (t10);


  \end{tikzpicture}
  \caption{Réseau de petri associé à l'exercice 8.1} \label{fig:M5}
\end{figure}

\newpage

\subsection{Question 2}
 Le protocole peut être décrit par le réseau suivant : \\

\begin{figure}[H]
  \centering
  \begin{tikzpicture}

    % Liste des places
    \draw (0,10) node[below right = 4pt] {$P_1$};
    \node[draw,circle,scale=2] (P1) at (0, 10) {};
    \draw (0,8) node[below right = 4pt] {$P_2$};
    \node[draw,circle,scale=2] (P2) at (0,8) {};
    \draw (0,6) node[below right = 4pt] {$P_3$};
    \node[draw,circle,scale=2] (P3) at (0, 6) {};
    \draw (-2,4) node[below right = 4pt] {$P_4$};
    \node[draw,circle,scale=2] (P4) at (-2,4) {};
    \draw (-2,2) node[below right = 4pt] {$P_5$};
    \node[draw,circle,scale=2] (P5) at (-2, 2) {};
     \draw (2,4) node[below right = 4pt] {$P_6$};
    \node[draw,circle,scale=2] (P6) at (2, 4) {};
    \draw (2,2) node[below right = 4pt] {$P_7$};
    \node[draw,circle,scale=2] (P7) at (2,2) {};
    \draw (4,2) node[below right = 4pt] {$P_8$};
    \node[draw,circle,scale=2] (P8) at (4, 2) {};
    \draw (2,0) node[below right = 4pt] {$P_9$};
    \node[draw,circle,scale=2] (P9) at (2,0) {};
    \draw (2,-2) node[below right = 4pt] {$P_10$};
    \node[draw,circle,scale=2] (P10) at (2,-2) {};

     % Liste des transitions
    \draw (0,9) node[below right = 4pt] {$t_1$};
    \node[draw,rectangle,yscale=2] (t1) at (0, 9) {};
    \draw (0,7) node[below right = 4pt] {$t_2$};
    \node[draw,rectangle,yscale=2] (t2) at (0, 7) {};
    \draw (-2,5) node[below right= 4pt] {$t_3$};
    \node[draw,rectangle,yscale=2] (t3) at (-2, 5) {};
    \draw (-2,3) node[below right = 4pt] {$t_4$};
    \node[draw,rectangle,yscale=2] (t4) at (-2, 3) {};
    \draw (-2,1) node[below = 10pt] {$t_5$};
    \node[draw,rectangle,yscale=2] (t5) at (-2, 1) {};
    \draw (2,5) node[below right = 3pt] {$t_6$};
    \node[draw,rectangle,yscale=2] (t6) at (2, 5) {};
    \draw (2,3) node[below right = 3pt] {$t_7$};
    \node[draw,rectangle,yscale=2] (t7) at (2, 3) {};
    \draw (4,3) node[below right = 3pt] {$t_8$};
    \node[draw,rectangle,yscale=2] (t8) at (4, 3) {};
    \draw (2,1) node[below right = 3pt] {$t_9$};
    \node[draw,rectangle,yscale=2] (t9) at (2, 1) {};
    \draw (4,1) node[below right = 3pt] {$t_10$};
    \node[draw,rectangle,yscale=2] (t10) at (4, 1) {};
    \draw (2,-1) node[below right = 10pt] {$t_11$};
    \node[draw,rectangle,yscale=2] (t11) at (2, -1) {};

     % Liste des arcs
    \draw[->,>=latex] (P1) -- (t1);
    \draw[->,>=latex] (t1) -- (P2);
    \draw[->,>=latex] (P2) -- (t2);
    \draw[->,>=latex] (t2) -- (P3);
    \draw[->,>=latex] (P3) -- (t3);
    \draw[->,>=latex] (t3) -- (P4);
    \draw[->,>=latex] (P4) -- (t4);
    \draw[->,>=latex] (t4) -- (P5);
    \draw[->,>=latex] (P5) -- (t5);
    \draw[->,>=latex] (t5) -- (P10);
    \draw[->,>=latex] (P3) -- (t6);
    \draw[->,>=latex] (t6) -- (P6);
    \draw[->,>=latex] (P6) -- (t7);
    \draw[->,>=latex] (P6) -- (t8);
    \draw[->,>=latex] (t7) -- (P7);
    \draw[->,>=latex] (t8) -- (P8);
    \draw[->,>=latex] (P7) -- (t9);
    \draw[->,>=latex] (t9) -- (P9);
    \draw[->,>=latex] (P9) -- (t11);
    \draw[->,>=latex] (t11) -- (P10);
    \draw[->,>=latex] (P8) -- (t10);
    \draw[->,>=latex] (t10) -- (P10);




  \end{tikzpicture}
  \caption{Réseau de petri associé à l'exercice 8.2} \label{fig:M5}
\end{figure}

\subsection{Question 3}
Le reseau  le plus adapté semble être le reseau numéro 2 pour plusieurs raisons : \\
    - Dans le reseau numero 1, la transition t1 est tous le temps franchissable donc on peut créer une infinité de construction sans savoir quand elle va finir.\\
    - Dans le reseau numéro 2, on met un nombre de jetons dans la place P1 correspondant au nombre de construction,
     elle sera terminé quand les jetons se retrouvent dans la place P10.    
